\documentclass{article}
\usepackage{graphicx}
\usepackage{hyperref}

%\usepackage{tikz}
\usepackage[default]{lato}
\usepackage[T1]{fontenc}
\usepackage{titlesec}
\usepackage{titling}
\usepackage[hmargin=0.5in,bmargin=1in,tmargin=1in,centering]{geometry}
\usetikzlibrary{shadows.blur}
\usetikzlibrary{shapes.symbols}
\usetikzlibrary{positioning,fit,hobby}

% The red used throughout the document
\definecolor{red}{HTML}{D32D2D}
\pagenumbering{gobble}  

% Defines the red lines at the top and bottom of each page
\newcommand\AddLines{%
\begin{tikzpicture}[remember picture,overlay]
    \fill[red] (current page.north west) rectangle ([yshift=-2mm]current page.north east);
    \fill[red] (current page.south west) rectangle ([yshift=2mm]current page.south east);
    \end{tikzpicture}%
}

% Adds the red lines to each page
\AtBeginShipout{\AddLines}
\AtBeginShipoutFirst{\AddLines}

% Defines the red block for section titles
\newcommand\SecTitle[1]{%
\begin{tikzpicture}
  \node[inner ysep=1cm,text width=\paperwidth,fill=red,text=white,font=\Huge]  at (0,0) 
 {\parbox{0.4in}{\mbox{}}\parbox{\dimexpr\textwidth-0.4in\relax}{\raggedright\strut#1\strut}\parbox{0pt}{\mbox{}}};
\end{tikzpicture}%
}

% Adds the red block to section titles
\titleformat{\section}{\normalfont}{}{-0.5in}{\SecTitle}
\title{Computational Graphing}

\begin{document}
    %\newcommand{\titleimage}{compiax.png}
    %\tikzstyle{title}=[font=\fontsize{25}{144}\selectfont, text width = \textwidth, align=left]
\tikzstyle{subtitle}=[font=\fontsize{20}{144}\selectfont, red, text width = \textwidth, align=right]
\tikzstyle{nam}=[font=\fontsize{15}{144}\selectfont, anchor=base, gray]
\tikzstyle{disclaimer}=[font=\fontsize{10}{144}\selectfont, white]
%\begin{titlepage}        
        \begin{tikzpicture}[remember picture,overlay, anchor = west]
            
            % Header
            \node[title] (title) at (-0.5,-6) {\documentname};
            \draw[line width=0.75mm, red] ([yshift=-0.75cm]title.west) -- ([yshift=-0.75cm]title.east);
            \node[subtitle] at ([yshift=-1.5cm]title.west) {Project Odin};
            
            \node[nam] at ([yshift=-10cm]current page.center){Kyle Erwin};
            \node[nam] at ([yshift=-10.6cm]current page.center){Joshua Cilliers};
            \node[nam] at ([yshift=-11.2cm]current page.center){Jason van Hattum};
            \node[nam] at ([yshift=-11.8cm]current page.center){Dimpho Mahoko};
            \node[nam] at ([yshift=-12.4cm]current page.center){Keegan Ferrett};
            
            % Logos
            \node at (-0.8,-21.5)
            {\includegraphics[width=4cm]{../Common/UniversityOfPretoriaLogo.png}};
            \node at (15,-22.0) {\includegraphics[width=5cm]{../Common/AlbertPrimeLogo.png}};  
            
            % red square at bottom of page
            \fill[red] (current page.south west) rectangle ([yshift=10mm]current page.south east);
          	% disclaimer
            \node[disclaimer] at ([yshift=5mm, xshift=2mm]current page.south west){Note: This document is constantly under revision due to our chosen methodology, and is subject to change.};
            \node[disclaimer] at ([yshift=5mm, xshift=-3.7cm]current page.south east){Current version: v\textbf{\documentversion}};
            
        \end{tikzpicture}        
%    \end{titlepage}
    
    %% Styles
\tikzstyle{table_number}=[white,font=\fontsize{80}{144}\selectfont]
\tikzstyle{table_heading}=[font=\fontsize{35}{144}\selectfont,text width = 14cm]

\begin{tikzpicture}[remember picture,overlay, anchor = north]
    % Clear footer and header
    \fill[white] ([xshift=5.8cm]current page.north west) rectangle ([yshift=-3mm]current page.north east);
    \fill[white] ([xshift=5.8cm]current page.south west) rectangle ([yshift=3mm]current page.south east);   
    % Sidebar and Title
    \fill[red] (current page.north west) rectangle ([xshift=5.8cm]current page.south west);
    \node[title] at (7,0) {Table of Contents};         
\end{tikzpicture}
    % Entries in the contents page
    %\begin{tikzpicture}[remember picture,overlay,anchor = north]
        % Project Overview
    %    \node[table_number] at (2.5,-3) {2};
    %    \node[table_heading] at (10.5,-3.5) {Project Overview};
        % Methodologies
    %    \node[table_number] at (2.5,-6) {3};
    %    \node[table_heading] at (10.5,-6.5) {Methodologies};
        % The Team
    %    \node[table_number] at (2.5,-9) {4};
    %    \node[table_heading] at (10.5,-9.5) {The Team}; 
    %\end{tikzpicture}
	
	\pagenumbering{arabic}

	\newpage

	\section{Project Overview}

	Our vision of this project is a very easy to use, web based, interpreted programming language. We see this tool as a way for people, 
    who may not be able to code, to write there own, short mathematical programs. It is important that this part of the tool be treated as
    any interpreter program would. Implementing typical interpreter components, such and lexical analysis and syntax analysis, is important
    to ensure that the libraries and packages, written by the user, preform in a reliable and consistant manner, such that the user can 
    trust whatever output their functions return. Our plan is try and write the interpreter in such a way that it can be run in either python
    or javascript on the client's side, to ensure that output is produced and displayed as fast as possible, as well as remove pressure 
    from the server. However, we do see this interpreter possibly getting to larger to pass through to the client, depending on what 
    funcationality is already implemented for the user (mathimatical operations, control structures, etc.), so running the interpreter on the
    server side may be the best option inorder to keep the web tool light weight. \par

    It is also extremely important that the web tool is easy to learn and use, as it will be used primarily by people who do not necessarily
    understand computer programming. We do not want to over whelm new users with too many options that they may not understand, but at the 
    same time we do want to provide them with all the tools that they could possibly need to create any function, as well as provide them 
    with any all the information, about their libraries, that they could need. This is where we see UI design being important. We will need 
    to create the website in such as way that users will be able to find whatever they may want easily and without getting confused and 
    frustated. Our team fully believes in good, intuitive GUI design that will make the users feeling comfortable when using our software.
    We would like to make use of a popular html/css framework, such as materialize css, to give the final product a very clean and modern look,
    well also speeding up our production time. We would also plan to do multiple usability tests towards the end of our development, to ensure
    that potential users will feel comfortable, and that we have succeeded in making the application intuitive for the users. \par

    It is important that all user information be stored in a safe an effient manner on the back-end, and that information retrieval be fast. 
    That is why we propose making use of the MEAN stack when creating the server and database. The MEAN stack will allow us to create a fast, 
    custom, webserver to handle and request that the user may have. Such requests would include, logging into the users account, fetching 
    libraries that the user may currently be working on, exploring other users libraries and profiles, etc. A mongo database will allow for 
    extremely fast data retrieval allowing the server to respond to more requests in a shorter time than a LAMP stack implementation. A MEAN 
    stack implemenation will allow us create a seperate gateway for exnternal components to access the libraries created by uses, as mentioned
    in the orignal brief.\par

    A possible potential feature that we would like to add to this tool, is the ability to download libraries and packages as a java/c++/python 
    class. The user would simply create and test the library on the web application and then select an option to download it as a prograaming
    language class. The library would then be sent to the node.js server, which would translate the graph into a class, and return the file to 
    the user. Users could also download any public libraries that someone else has implemented, to use in their own programs. The would not be 
    such a tough feature to implement if we treat this application as an interpreter, as we will just be translating code. This feature will 
    allow programmers the ability to quickly write and test mathematical functions which they can the use in their programs. This could help 
    attract more users which a different skill set to the site. \\
    \\
    \noindent\textbf{The following is our proposed deployment of the system} \\
    \includegraphics[width=\textwidth]{deployment}

	\newpage
	\section{Proposed Methodology}
	We value the relationship formed between the client and our team and the importance of having a good relationship. So much in fact that we would want to include throught the whole process of building your project by presenting demos and working with you on the feedback; letting you know how far we are from completion and what we have accomplished over each week. This all forms just one part of the methodology we've chosen to development your project. The agile method. Specifically, feature driven development. One of the great things about the agile methods is that it allows us to respond to change very quickly. So if you have any ideas halfway through the process we will be able to incorporate them into the final product in one form or another. We will begin by developing an overall model of the system. The model will represent our solution and how we intend to develop it. Once we have agreed upon the system model we can begin working on the feature list. This list will contain all the features you wish your project to have. We will ranks each as either a major or minor feature and begin working through such accordingly. For each feature we intend to develop a plan to construct the feature. Latsly, develop the feature. As we go through we'll go through each feature on the feature list we'll create the plan, develop the feature and move to the next feature. We believe this will provide the best experience for communication and producing your product.\\
\\
We would like to meet our clients as soon as possible to begin discussing your vision for the project and to clarrify as much as possible
before we begin work. Currently there are 3 demos assigned for this project:
	\begin{itemize}
        \item Demo 1: 26th May
        \item Demo 2: 28th July
        \item Demo 3: 1st September
    \end{itemize}
During these demos we will show you the progress we have made and get feedback from you about what you like and what you would want changed.
Our current plan for the demo meetings are as follows:
	\begin{itemize}
        \item Demo 1: Dicuss requirement documentation that we have produced, as well as demo a mock front-end that we have produced to 
        demonstrate how these requirements can be met. 
        \item Demo 2: Dicuss design documentation that we have produced, as well as demo the progress we have made with the various subsystems 
        of the project.
        \item Demo 3: Demo the various subsystems of the project, and potentially have a working, intergrated prototype of the full system, as 
        well as present some user doumentation.
    \end{itemize}
During each of these demo sessions, we would apperiate any feedback that you may have. Any criticisms or advice that you may have for us
will be greatly appreciated, as we greatly value your input and believe that it is important inorder to deliver the product that you require.
During the final evaluation phase, which begins on the 13th October, our client will recieve all of documentation as well as a fully 
intergrated system.\\
\\ 
Please note that, as the client, you are more than welcome to adjust this timetable as you see fit. Additionally if you would like to have any 
additional meetings to check our progress, or make an adjustment to the specification, we would be more than happy to arrange it. We believe 
the more input we get from you as a client, then more refined the final product will be.
	\newpage
	\section{The Team}
	\textbf{Dorothy Mahoko} \\
Lorem ipsum dolor sit amet, consectetur adipiscing elit. Curabitur aliquam augue a odio cursus bibendum. Suspendisse felis diam, varius eu molestie quis, condimentum eget libero. Fusce egestas ligula sit amet metus vehicula ornare. Integer et magna sapien. Pellentesque nec metus in sapien congue gravida eu quis dolor. Maecenas consequat nunc a enim ullamcorper venenatis. Suspendisse at faucibus dolor, imperdiet rhoncus ante.\\ \\
\textbf{Jason van Hattum}\\ 
Lorem ipsum dolor sit amet, consectetur adipiscing elit. Curabitur aliquam augue a odio cursus bibendum. Suspendisse felis diam, varius eu molestie quis, condimentum eget libero. Fusce egestas ligula sit amet metus vehicula ornare. Integer et magna sapien. Pellentesque nec metus in sapien congue gravida eu quis dolor. Maecenas consequat nunc a enim ullamcorper venenatis. Suspendisse at faucibus dolor, imperdiet rhoncus ante.\\ \\
\textbf{Kyle Erwin}\\
Lorem ipsum dolor sit amet, consectetur adipiscing elit. Curabitur aliquam augue a odio cursus bibendum. Suspendisse felis diam, varius eu molestie quis, condimentum eget libero. Fusce egestas ligula sit amet metus vehicula ornare. Integer et magna sapien. Pellentesque nec metus in sapien congue gravida eu quis dolor. Maecenas consequat nunc a enim ullamcorper venenatis. Suspendisse at faucibus dolor, imperdiet rhoncus ante.\\ \\
\textbf{Joshua Cilliers}\\
Lorem ipsum dolor sit amet, consectetur adipiscing elit. Curabitur aliquam augue a odio cursus bibendum. Suspendisse felis diam, varius eu molestie quis, condimentum eget libero. Fusce egestas ligula sit amet metus vehicula ornare. Integer et magna sapien. Pellentesque nec metus in sapien congue gravida eu quis dolor. Maecenas consequat nunc a enim ullamcorper venenatis. Suspendisse at faucibus dolor, imperdiet rhoncus ante. \\ \\
\textbf{Keegan Ferrett}\\
Lorem ipsum dolor sit amet, consectetur adipiscing elit. Curabitur aliquam augue a odio cursus bibendum. Suspendisse felis diam, varius eu molestie quis, condimentum eget libero. Fusce egestas ligula sit amet metus vehicula ornare. Integer et magna sapien. Pellentesque nec metus in sapien congue gravida eu quis dolor. Maecenas consequat nunc a enim ullamcorper venenatis. Suspendisse at faucibus dolor, imperdiet rhoncus ante.

	\section{Why Albert Prime?}

    Two of our memembers have both completed the compiler construction course at the University of Pretoria, and will be able to apply 
    the knowledge gained from building a compiler into building the necessary components needed for the interpreter component of this project. 
    It is important that this is done correctly to ensure that no unexpected errors or crashes occur when a user is testing their 
    created library. Although this may be seen as a very basic interpreter, it will still be important to implement all the necessary 
    components of an interpreter/compiler to ensure that, even as functions and projects get larger, the user can still be sure that they are
    getting reliable and consistant results, without running the risk of potentially crashing the node.js server. It will also give us the 
    ability of being able to add more basic, such as if statements and loops, with little effort.\par

    Every member of our team has experience with working on front-end and back-end web development. This is necessary for this project 
    as our vision for the final product would involve much of what the user is seeing, being done on the client's side, as well as having
    admistrative tasks, being worked on on the back-end. The front-end needs aesthetically pleasing as well as being easy to use and intuitive,
    well the back-end needs to be secure, safe, and reliable to ensure that no personal data is lost and that the web server stays up 
    and running at all times. Each member has experience working with the MEAN stack giving us the advantage of not needing to relearn the
    the thechnologies used in a MEAN stack implementation.\par

    Finally, each member of our team is a skilled mathematician, having completed calculus, linear algebra, and discrete mathematics courses.
    This means we will not have any trouble implementing and testing the mathematical logic. We will be able to account of any possible need
    that a user may have, such as binary mathematics, discrete mathematics, or linear algebra operations.


\end{document}