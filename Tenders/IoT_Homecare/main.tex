\documentclass{article}
\usepackage{graphicx}
\usepackage{hyperref}

\usepackage{tikz}
\usepackage[default]{lato}
\usepackage[T1]{fontenc}
\usepackage{titlesec}
\usepackage{titling}
\usepackage[hmargin=0.5in,bmargin=1in,tmargin=1in,centering]{geometry}
\usetikzlibrary{shadows.blur}
\usetikzlibrary{shapes.symbols}
\usetikzlibrary{positioning,fit,hobby}

% The red used throughout the document
\definecolor{red}{HTML}{D32D2D}
\pagenumbering{gobble}  

% Defines the red lines at the top and bottom of each page
\newcommand\AddLines{%
\begin{tikzpicture}[remember picture,overlay]
    \fill[red] (current page.north west) rectangle ([yshift=-2mm]current page.north east);
    \fill[red] (current page.south west) rectangle ([yshift=2mm]current page.south east);
    \end{tikzpicture}%
}

% Adds the red lines to each page
\AtBeginShipout{\AddLines}
\AtBeginShipoutFirst{\AddLines}

% Defines the red block for section titles
\newcommand\SecTitle[1]{%
\begin{tikzpicture}
  \node[inner ysep=1cm,text width=\paperwidth,fill=red,text=white,font=\Huge]  at (0,0) 
 {\parbox{0.4in}{\mbox{}}\parbox{\dimexpr\textwidth-0.4in\relax}{\raggedright\strut#1\strut}\parbox{0pt}{\mbox{}}};
\end{tikzpicture}%
}

% Adds the red block to section titles
\titleformat{\section}{\normalfont}{}{-0.5in}{\SecTitle}
\title{IOT Homecare}

\begin{document}
    \newcommand{\titleimage}{iot.png}
    \tikzstyle{title}=[font=\fontsize{25}{144}\selectfont, text width = \textwidth, align=left]
\tikzstyle{subtitle}=[font=\fontsize{20}{144}\selectfont, red, text width = \textwidth, align=right]
\tikzstyle{nam}=[font=\fontsize{15}{144}\selectfont, anchor=base, gray]
\tikzstyle{disclaimer}=[font=\fontsize{10}{144}\selectfont, white]
%\begin{titlepage}        
        \begin{tikzpicture}[remember picture,overlay, anchor = west]
            
            % Header
            \node[title] (title) at (-0.5,-6) {\documentname};
            \draw[line width=0.75mm, red] ([yshift=-0.75cm]title.west) -- ([yshift=-0.75cm]title.east);
            \node[subtitle] at ([yshift=-1.5cm]title.west) {Project Odin};
            
            \node[nam] at ([yshift=-10cm]current page.center){Kyle Erwin};
            \node[nam] at ([yshift=-10.6cm]current page.center){Joshua Cilliers};
            \node[nam] at ([yshift=-11.2cm]current page.center){Jason van Hattum};
            \node[nam] at ([yshift=-11.8cm]current page.center){Dimpho Mahoko};
            \node[nam] at ([yshift=-12.4cm]current page.center){Keegan Ferrett};
            
            % Logos
            \node at (-0.8,-21.5)
            {\includegraphics[width=4cm]{../Common/UniversityOfPretoriaLogo.png}};
            \node at (15,-22.0) {\includegraphics[width=5cm]{../Common/AlbertPrimeLogo.png}};  
            
            % red square at bottom of page
            \fill[red] (current page.south west) rectangle ([yshift=10mm]current page.south east);
          	% disclaimer
            \node[disclaimer] at ([yshift=5mm, xshift=2mm]current page.south west){Note: This document is constantly under revision due to our chosen methodology, and is subject to change.};
            \node[disclaimer] at ([yshift=5mm, xshift=-3.7cm]current page.south east){Current version: v\textbf{\documentversion}};
            
        \end{tikzpicture}        
%    \end{titlepage}
    
    % Styles
\tikzstyle{table_number}=[white,font=\fontsize{80}{144}\selectfont]
\tikzstyle{table_heading}=[font=\fontsize{35}{144}\selectfont,text width = 14cm]

\begin{tikzpicture}[remember picture,overlay, anchor = north]
    % Clear footer and header
    \fill[white] ([xshift=5.8cm]current page.north west) rectangle ([yshift=-3mm]current page.north east);
    \fill[white] ([xshift=5.8cm]current page.south west) rectangle ([yshift=3mm]current page.south east);   
    % Sidebar and Title
    \fill[red] (current page.north west) rectangle ([xshift=5.8cm]current page.south west);
    \node[title] at (7,0) {Table of Contents};         
\end{tikzpicture}
    % Entries in the contents page
    \begin{tikzpicture}[remember picture,overlay,anchor = north]
        % Project Overview
        \node[table_number] at (2.5,-3) {2};
        \node[table_heading] at (10.5,-3.5) {Project Overview};
        % Methodologies
        \node[table_number] at (2.5,-6) {3};
        \node[table_heading] at (10.5,-6.5) {Methodologies};
        % The Team
        \node[table_number] at (2.5,-9) {4};
        \node[table_heading] at (10.5,-9.5) {The Team}; 
    \end{tikzpicture}
	
	\pagenumbering{arabic}

	\newpage

	\section{Project Overview}

	As an elderly or just sickly person, it is common for those who can afford it to hire a 24 hour caregiver to monitor the patient and give them attention when needed. This becomes a problem when you cannot afford the service or you're just uncomfortable with introducing this strange person into your life. The Internet Of Things HomeCare System is a system designed to reduce costs of being a sick person such that you do not need a 24 hour caregiver with you checking your blood sugar levels, heart-rate and what not every 2 hours. \\
	
	The name we have chosen for the system is Nightingale because it represents a guardian which is what the system in question is basically supposed to do.\\
	
	A few ideas for how the system can be implemented and used: \\
	
	\begin{itemize}
		
		\item An automated medicine dispenser that can be set to dispense medicine according to time and dosage of the person it is meant to take care of. The dispenser can also have a reminder each time it is time to take the medicine. All the tablets can be dispensed into one small bowl so that the patient doesn't have to keep track themselves.\\
		
		\item The system can automatically turn the sprinklers on and off depending on time and weather such that if it's going to rain, water won't be wasted. This can be applied to household items like curtains, windows, lights etc such that if there is no movement in a room for a certain period of time the lights turn off and on again if there is motion. \\
		
		
		
		\item The system can include room sensors that can notify the caregivers when there is an unusual pattern of movement. If the patient hasn't moved in an unreasonable amount of time, taking sleeping and "when out" into consideration.\\
		
		
		\item We can include a smart bracelet that monitors the user's heart rate. The bracelet can include a panic button that the user can press should anything need immediate attention, signaling the caregiver. The heart rate monitor can send data periodically or on request. \\
		
		\item A feature that can create an alert if the stove has been on and empty, the tap running or the lights on for an unreasonable amount of time let the user know incase they've forgotten to prevent harm and save resources.\\
		
		\item The system will include an application that can be used by the caregiver to monitor the user and if need be, control the devices and sensors. The patient can also use the mobile application to control the devices and sensors but certain functions like changing medical dosages and switching sensors off will require administrative authorization. This can be used to control devices around the house such as the TV, lights, dishwasher etc.\\
		
		\item The application will also include a login for the family of the patient where they can check up on the user and communicate with the caregiver for updates.  \\
		
		This system is one that has no boundaries. The innovative possibilities are endless. The implementation will have to be modular because we need to be able to add features over time and make very few to no changes to the system itself.
		
	\end{itemize}

	\newpage
	\section{Proposed Methodology}
	We value the relationship formed between the client and our team and the importance of having a good relationship. So much in fact that we would want to include throught the whole process of building your project by presenting demos and working with you on the feedback; letting you know how far we are from completion and what we have accomplished over each week. This all forms just one part of the methodology we've chosen to development your project. The agile method. Specifically, feature driven development. One of the great things about the agile methods is that it allows us to respond to change very quickly. So if you have any ideas halfway through the process we will be able to incorporate them into the final product in one form or another. We will begin by developing an overall model of the system. The model will represent our solution and how we intend to develop it. Once we have agreed upon the system model we can begin working on the feature list. This list will contain all the features you wish your project to have. We will ranks each as either a major or minor feature and begin working through such accordingly. For each feature we intend to develop a plan to construct the feature. Latsly, develop the feature. As we go through we'll go through each feature on the feature list we'll create the plan, develop the feature and move to the next feature. We believe this will provide the best experience for communication and producing your product.\\
\\
We would like to meet our clients as soon as possible to begin discussing your vision for the project and to clarrify as much as possible
before we begin work. Currently there are 3 demos assigned for this project:
	\begin{itemize}
        \item Demo 1: 26th May
        \item Demo 2: 28th July
        \item Demo 3: 1st September
    \end{itemize}
During these demos we will show you the progress we have made and get feedback from you about what you like and what you would want changed.
Our current plan for the demo meetings are as follows:
	\begin{itemize}
        \item Demo 1: Dicuss requirement documentation that we have produced, as well as demo a mock front-end that we have produced to 
        demonstrate how these requirements can be met. 
        \item Demo 2: Dicuss design documentation that we have produced, as well as demo the progress we have made with the various subsystems 
        of the project.
        \item Demo 3: Demo the various subsystems of the project, and potentially have a working, intergrated prototype of the full system, as 
        well as present some user doumentation.
    \end{itemize}
During each of these demo sessions, we would apperiate any feedback that you may have. Any criticisms or advice that you may have for us
will be greatly appreciated, as we greatly value your input and believe that it is important inorder to deliver the product that you require.
During the final evaluation phase, which begins on the 13th October, our client will recieve all of documentation as well as a fully 
intergrated system.\\
\\ 
Please note that, as the client, you are more than welcome to adjust this timetable as you see fit. Additionally if you would like to have any 
additional meetings to check our progress, or make an adjustment to the specification, we would be more than happy to arrange it. We believe 
the more input we get from you as a client, then more refined the final product will be.
	\newpage
	\section{The Team}
	\textbf{Dorothy Mahoko} \\
Lorem ipsum dolor sit amet, consectetur adipiscing elit. Curabitur aliquam augue a odio cursus bibendum. Suspendisse felis diam, varius eu molestie quis, condimentum eget libero. Fusce egestas ligula sit amet metus vehicula ornare. Integer et magna sapien. Pellentesque nec metus in sapien congue gravida eu quis dolor. Maecenas consequat nunc a enim ullamcorper venenatis. Suspendisse at faucibus dolor, imperdiet rhoncus ante.\\ \\
\textbf{Jason van Hattum}\\ 
Lorem ipsum dolor sit amet, consectetur adipiscing elit. Curabitur aliquam augue a odio cursus bibendum. Suspendisse felis diam, varius eu molestie quis, condimentum eget libero. Fusce egestas ligula sit amet metus vehicula ornare. Integer et magna sapien. Pellentesque nec metus in sapien congue gravida eu quis dolor. Maecenas consequat nunc a enim ullamcorper venenatis. Suspendisse at faucibus dolor, imperdiet rhoncus ante.\\ \\
\textbf{Kyle Erwin}\\
Lorem ipsum dolor sit amet, consectetur adipiscing elit. Curabitur aliquam augue a odio cursus bibendum. Suspendisse felis diam, varius eu molestie quis, condimentum eget libero. Fusce egestas ligula sit amet metus vehicula ornare. Integer et magna sapien. Pellentesque nec metus in sapien congue gravida eu quis dolor. Maecenas consequat nunc a enim ullamcorper venenatis. Suspendisse at faucibus dolor, imperdiet rhoncus ante.\\ \\
\textbf{Joshua Cilliers}\\
Lorem ipsum dolor sit amet, consectetur adipiscing elit. Curabitur aliquam augue a odio cursus bibendum. Suspendisse felis diam, varius eu molestie quis, condimentum eget libero. Fusce egestas ligula sit amet metus vehicula ornare. Integer et magna sapien. Pellentesque nec metus in sapien congue gravida eu quis dolor. Maecenas consequat nunc a enim ullamcorper venenatis. Suspendisse at faucibus dolor, imperdiet rhoncus ante. \\ \\
\textbf{Keegan Ferrett}\\
Lorem ipsum dolor sit amet, consectetur adipiscing elit. Curabitur aliquam augue a odio cursus bibendum. Suspendisse felis diam, varius eu molestie quis, condimentum eget libero. Fusce egestas ligula sit amet metus vehicula ornare. Integer et magna sapien. Pellentesque nec metus in sapien congue gravida eu quis dolor. Maecenas consequat nunc a enim ullamcorper venenatis. Suspendisse at faucibus dolor, imperdiet rhoncus ante.
	\newpage
	\section{Why Albert Prime?}
	
	Our team consists of a set of people with enough skills to complete this project sufficiently and thensome. We are all sufficiently competent in MEAN stack. \\ 
	
	Keegan is our resident Raspberry Pi expert. He has a passion for Raspberry Pi and works with them on a regular basis. This will aid the development of this system as one of the requirements is to program a Raspberry Pi to gather the data from the devices around the house.\\
	
	As third year Computer Science Students, we know a number of programming languages. It is worth mentioning that Jason is very fluent in Python amongst other languages. It will be an asset to your system to have someone with his expertise working on it.\\
	
	
	Application development is one of our strong points. The mobile application needed for the caretaker communication and control of devices will make use of our skills. Joshua and Keegan's experience with iOS and Android development and his passion for mobile application development amongst others will be put to great use with this system.\\
	
	This system will require creativity and and a good understanding of cloud servers. The Rabbiteer program Dimpho attended in 2016 focused a lot on this. This project will be an opportunity to develop these skills and her success in the Standard bank IT challenge shows her creative capability and competence.\\
	
	Kyle is very passionate about Artificial Intelligence which is always a useful skill in today's age. This system could use his passion and skill for the monitoring and learning the user's patterns such that if anything seems out of place, the caregivers can notified.\\
	
	All 5 group members are talented developers. Above are a few of the skills we have to offer but our skills are not limited to what is mentioned above. We are all hardworking fast learners and willing to learn any new technologies and skills needed to make this system a successful one. 
	\newpage

	%Give description on why we are good for this project


\end{document}